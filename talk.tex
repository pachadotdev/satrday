\documentclass[ignorenonframetext,compress,aspectratio=169]{beamer}
\setbeamertemplate{caption}[numbered]
\setbeamertemplate{caption label separator}{: }
\setbeamercolor{caption name}{fg=normal text.fg}
\beamertemplatenavigationsymbolsempty
\usepackage{lmodern}
\usepackage{amssymb,amsmath}
\usepackage{ifxetex,ifluatex}
\usepackage{fixltx2e} % provides \textsubscript
\ifnum 0\ifxetex 1\fi\ifluatex 1\fi=0 % if pdftex
  \usepackage[T1]{fontenc}
  \usepackage[utf8]{inputenc}
\else % if luatex or xelatex
  \ifxetex
    \usepackage{mathspec}
  \else
    \usepackage{fontspec}
  \fi
  \defaultfontfeatures{Ligatures=TeX,Scale=MatchLowercase}
\fi
\usetheme[]{metropolis}
% use upquote if available, for straight quotes in verbatim environments
\IfFileExists{upquote.sty}{\usepackage{upquote}}{}
% use microtype if available
\IfFileExists{microtype.sty}{%
\usepackage{microtype}
\UseMicrotypeSet[protrusion]{basicmath} % disable protrusion for tt fonts
}{}
\newif\ifbibliography
\usepackage{natbib}
\bibliographystyle{plainnat}
\hypersetup{
            pdftitle={Econometrics in R: Estimation of Gravity Models},
            pdfauthor={Pachá},
            pdfborder={0 0 0},
            breaklinks=true}
\urlstyle{same}  % don't use monospace font for urls
\usepackage{color}
\usepackage{fancyvrb}
\newcommand{\VerbBar}{|}
\newcommand{\VERB}{\Verb[commandchars=\\\{\}]}
\DefineVerbatimEnvironment{Highlighting}{Verbatim}{commandchars=\\\{\}}
% Add ',fontsize=\small' for more characters per line
\usepackage{framed}
\definecolor{shadecolor}{RGB}{248,248,248}
\newenvironment{Shaded}{\begin{snugshade}}{\end{snugshade}}
\newcommand{\KeywordTok}[1]{\textcolor[rgb]{0.13,0.29,0.53}{\textbf{#1}}}
\newcommand{\DataTypeTok}[1]{\textcolor[rgb]{0.13,0.29,0.53}{#1}}
\newcommand{\DecValTok}[1]{\textcolor[rgb]{0.00,0.00,0.81}{#1}}
\newcommand{\BaseNTok}[1]{\textcolor[rgb]{0.00,0.00,0.81}{#1}}
\newcommand{\FloatTok}[1]{\textcolor[rgb]{0.00,0.00,0.81}{#1}}
\newcommand{\ConstantTok}[1]{\textcolor[rgb]{0.00,0.00,0.00}{#1}}
\newcommand{\CharTok}[1]{\textcolor[rgb]{0.31,0.60,0.02}{#1}}
\newcommand{\SpecialCharTok}[1]{\textcolor[rgb]{0.00,0.00,0.00}{#1}}
\newcommand{\StringTok}[1]{\textcolor[rgb]{0.31,0.60,0.02}{#1}}
\newcommand{\VerbatimStringTok}[1]{\textcolor[rgb]{0.31,0.60,0.02}{#1}}
\newcommand{\SpecialStringTok}[1]{\textcolor[rgb]{0.31,0.60,0.02}{#1}}
\newcommand{\ImportTok}[1]{#1}
\newcommand{\CommentTok}[1]{\textcolor[rgb]{0.56,0.35,0.01}{\textit{#1}}}
\newcommand{\DocumentationTok}[1]{\textcolor[rgb]{0.56,0.35,0.01}{\textbf{\textit{#1}}}}
\newcommand{\AnnotationTok}[1]{\textcolor[rgb]{0.56,0.35,0.01}{\textbf{\textit{#1}}}}
\newcommand{\CommentVarTok}[1]{\textcolor[rgb]{0.56,0.35,0.01}{\textbf{\textit{#1}}}}
\newcommand{\OtherTok}[1]{\textcolor[rgb]{0.56,0.35,0.01}{#1}}
\newcommand{\FunctionTok}[1]{\textcolor[rgb]{0.00,0.00,0.00}{#1}}
\newcommand{\VariableTok}[1]{\textcolor[rgb]{0.00,0.00,0.00}{#1}}
\newcommand{\ControlFlowTok}[1]{\textcolor[rgb]{0.13,0.29,0.53}{\textbf{#1}}}
\newcommand{\OperatorTok}[1]{\textcolor[rgb]{0.81,0.36,0.00}{\textbf{#1}}}
\newcommand{\BuiltInTok}[1]{#1}
\newcommand{\ExtensionTok}[1]{#1}
\newcommand{\PreprocessorTok}[1]{\textcolor[rgb]{0.56,0.35,0.01}{\textit{#1}}}
\newcommand{\AttributeTok}[1]{\textcolor[rgb]{0.77,0.63,0.00}{#1}}
\newcommand{\RegionMarkerTok}[1]{#1}
\newcommand{\InformationTok}[1]{\textcolor[rgb]{0.56,0.35,0.01}{\textbf{\textit{#1}}}}
\newcommand{\WarningTok}[1]{\textcolor[rgb]{0.56,0.35,0.01}{\textbf{\textit{#1}}}}
\newcommand{\AlertTok}[1]{\textcolor[rgb]{0.94,0.16,0.16}{#1}}
\newcommand{\ErrorTok}[1]{\textcolor[rgb]{0.64,0.00,0.00}{\textbf{#1}}}
\newcommand{\NormalTok}[1]{#1}
\usepackage{graphicx,grffile}
\makeatletter
\def\maxwidth{\ifdim\Gin@nat@width>\linewidth\linewidth\else\Gin@nat@width\fi}
\def\maxheight{\ifdim\Gin@nat@height>\textheight0.8\textheight\else\Gin@nat@height\fi}
\makeatother
% Scale images if necessary, so that they will not overflow the page
% margins by default, and it is still possible to overwrite the defaults
% using explicit options in \includegraphics[width, height, ...]{}
\setkeys{Gin}{width=\maxwidth,height=\maxheight,keepaspectratio}

% Prevent slide breaks in the middle of a paragraph:
\widowpenalties 1 10000
\raggedbottom

\AtBeginPart{
  \let\insertpartnumber\relax
  \let\partname\relax
  \frame{\partpage}
}
\AtBeginSection{
  \ifbibliography
  \else
    \let\insertsectionnumber\relax
    \let\sectionname\relax
    \frame{\sectionpage}
  \fi
}
\AtBeginSubsection{
  \let\insertsubsectionnumber\relax
  \let\subsectionname\relax
  \frame{\subsectionpage}
}

\setlength{\parindent}{0pt}
\setlength{\parskip}{6pt plus 2pt minus 1pt}
\setlength{\emergencystretch}{3em}  % prevent overfull lines
\providecommand{\tightlist}{%
  \setlength{\itemsep}{0pt}\setlength{\parskip}{0pt}}
\setcounter{secnumdepth}{0}
\hypersetup{colorlinks,citecolor=orange,filecolor=red,linkcolor=brown,urlcolor=blue}
%\usepackage[semibold]{sourcesanspro}
\usepackage{fontawesome}
\renewcommand{\vec}[1]{\boldsymbol{#1}}
\newcommand{\R}{\mathbb{R}}

\title{Econometrics in R: Estimation of Gravity Models}
\author{Pachá}
\institute{satRday Santiago}
\date{Dec 15, 2018}

\begin{document}
\frame{\titlepage}

\section{Before we begin}\label{before-we-begin}

\begin{frame}{This is about a new R package}

\begin{figure}
\centering
\includegraphics{hexicon.PNG}
\caption{Hex sticker}
\end{figure}

\end{frame}

\begin{frame}[fragile]{Where to reach me}

\faTwitter or \faGithub: pachamaltese

\faEnvelopeO: \texttt{m\ vargas\ at\ dcc\ dot\ uchile\ dot\ cl}

\faPhone: +1 XXX XXX XX XX or +56 X X XXX XX XX

\end{frame}

\begin{frame}{Question to the audience}

\LARGE{Do you understand (a bit) linear algebra?}

\LARGE{Have you ever fitted a regression (in R) before today?}

\end{frame}

\begin{frame}{Linear regression}

Let \(\vec{y} \in \R^n\) be the outcome and \(X \in \R^{n\times p}\) be
the design matrix in the context of a general model with intercept:
\[\vec{y} = X\vec{\beta} + \vec{e}\]

Being: \[
\underset{n\times 1}{\vec{y}} = \begin{pmatrix}y_0 \cr y_1 \cr \vdots \cr y_n\end{pmatrix}
\text{ and }
\underset{n\times p}{X} = \begin{pmatrix}1 & x_{11} & & x_{1p} \cr 1 & x_{21} & & x_{2p} \cr & \ddots & \cr 1 & x_{n1} & & x_{np}\end{pmatrix} = (\vec{1} \: \vec{x}_1 \: \ldots \: \vec{x}_p)
\]

\end{frame}

\begin{frame}{Linear regression}

In linear models the aim is to minimize the error term by chosing
\(\hat{\vec{\beta}}\). One possibility is to minimize the squared error
by solving this optimization problem:

\begin{equation}
\label{min}
\displaystyle \min_{\vec{\beta}} S = \|\vec{y} - X\vec{\beta}\|^2
\end{equation}

\end{frame}

\begin{frame}{Linear regression}

Books such as \citet{Baltagi2011} discuss how to solve \eqref{min} and
other equivalent approaches that result in this optimal estimator:

\begin{equation}
\label{beta}
\hat{\vec{\beta}} = (X^tX)^{-1} X^t\vec{y}
\end{equation}

\end{frame}

\begin{frame}{Linear regression}

With one independent variable and intercept, this is
\(y_i = \beta_0 + \beta_1 x_{i1} + e_i\), equation \(\eqref{beta}\)
means:

\begin
{equation}
\label{beta2}
\hat{\beta}_1 = cor(\vec{y},\vec{x}) \cdot \frac{sd(\vec{y})}{sd(\vec{x})} \text{ and } \hat{\beta}_0 = \bar{y} - \hat{\beta}_1 \bar{\vec{x}}
\end{equation}

\end{frame}

\begin{frame}[fragile]{Coding example with mtcars dataset}

Consider the model: \[mpg_i = \beta_1 wt_i + \beta_2 cyl_i + e_i\]

This is how to write that model in R notation:

\begin{Shaded}
\begin{Highlighting}[]
\KeywordTok{lm}\NormalTok{(mpg }\OperatorTok{~}\StringTok{ }\NormalTok{wt }\OperatorTok{+}\StringTok{ }\NormalTok{cyl, }\DataTypeTok{data =}\NormalTok{ mtcars)}
\end{Highlighting}
\end{Shaded}

\end{frame}

\begin{frame}[fragile]{Coding example with mtcars dataset}

Or written in matrix form:

\begin{Shaded}
\begin{Highlighting}[]
\NormalTok{y <-}\StringTok{ }\NormalTok{mtcars}\OperatorTok{$}\NormalTok{mpg}
\NormalTok{x0 <-}\StringTok{ }\KeywordTok{rep}\NormalTok{(}\DecValTok{1}\NormalTok{, }\KeywordTok{length}\NormalTok{(y))}
\NormalTok{x1 <-}\StringTok{ }\NormalTok{mtcars}\OperatorTok{$}\NormalTok{wt}
\NormalTok{x2 <-}\StringTok{ }\NormalTok{mtcars}\OperatorTok{$}\NormalTok{cyl}
\NormalTok{X <-}\StringTok{ }\KeywordTok{cbind}\NormalTok{(x0,x1,x2)}
\end{Highlighting}
\end{Shaded}

\end{frame}

\begin{frame}[fragile]{Coding example with mtcars dataset}

It's the same to use \texttt{lm} or to perform a matrix multiplication
because of equation \(\eqref{beta}\):

\begin{Shaded}
\begin{Highlighting}[]
\NormalTok{fit <-}\StringTok{ }\KeywordTok{lm}\NormalTok{(y }\OperatorTok{~}\StringTok{ }\NormalTok{x1 }\OperatorTok{+}\StringTok{ }\NormalTok{x2)}
\NormalTok{beta <-}\StringTok{ }\KeywordTok{solve}\NormalTok{(}\KeywordTok{t}\NormalTok{(X)}\OperatorTok\NormalTok{X) }\OperatorTok\StringTok{ }\NormalTok{(}\KeywordTok{t}\NormalTok{(X)}\OperatorTok\NormalTok{y)}
\end{Highlighting}
\end{Shaded}

\end{frame}

\begin{frame}[fragile]{Coding example with mtcars dataset}

It's the same to use \texttt{lm} or to perform a matrix multiplication
because of equation \(\eqref{beta}\):

\begin{Shaded}
\begin{Highlighting}[]
\KeywordTok{coefficients}\NormalTok{(fit)}
\end{Highlighting}
\end{Shaded}

\begin{verbatim}
## (Intercept)          x1          x2 
##   39.686261   -3.190972   -1.507795
\end{verbatim}

\begin{Shaded}
\begin{Highlighting}[]
\NormalTok{beta}
\end{Highlighting}
\end{Shaded}

\begin{verbatim}
##         [,1]
## x0 39.686261
## x1 -3.190972
## x2 -1.507795
\end{verbatim}

\end{frame}

\begin{frame}[fragile]{Coding example with Galton dataset}

Equation \(\eqref{beta2}\) can be verified with R commands:

\begin{Shaded}
\begin{Highlighting}[]
\ControlFlowTok{if}\NormalTok{ (}\OperatorTok{!}\KeywordTok{require}\NormalTok{(pacman)) }\KeywordTok{install.packages}\NormalTok{(}\StringTok{"pacman"}\NormalTok{)}
\KeywordTok{p_load}\NormalTok{(HistData)}

\NormalTok{y <-}\StringTok{ }\NormalTok{Galton}\OperatorTok{$}\NormalTok{child}
\NormalTok{x <-}\StringTok{ }\NormalTok{Galton}\OperatorTok{$}\NormalTok{parent}
\NormalTok{beta1 <-}\StringTok{ }\KeywordTok{cor}\NormalTok{(y, x) }\OperatorTok{*}\StringTok{  }\KeywordTok{sd}\NormalTok{(y) }\OperatorTok{/}\StringTok{ }\KeywordTok{sd}\NormalTok{(x)}
\NormalTok{beta0 <-}\StringTok{ }\KeywordTok{mean}\NormalTok{(y) }\OperatorTok{-}\StringTok{ }\NormalTok{beta1 }\OperatorTok{*}\StringTok{ }\KeywordTok{mean}\NormalTok{(x)}
\KeywordTok{c}\NormalTok{(beta0, beta1)}
\end{Highlighting}
\end{Shaded}

\begin{verbatim}
## [1] 23.9415302  0.6462906
\end{verbatim}

\end{frame}

\begin{frame}[fragile]{Coding example with Galton dataset}

\begin{verbatim}
## 
## Call:
## lm(formula = y ~ x)
## 
## Coefficients:
## (Intercept)            x  
##     23.9415       0.6463
\end{verbatim}

\end{frame}

\section{Simple gravity model}\label{simple-gravity-model}

\begin{frame}{Simple gravity model}

The main reference for this section is \citet{Woelver2018} and the
references therein.

Gravity models in their traditional form are inspired by Newton law of
gravitation: \[
F_{ij}=G\frac{M_{i}M_{j}}{D^{2}_{ij}}.
\]

The force \(F\) between two bodies \(i\) and \(j\) with \(i \neq j\) is
proportional to the masses \(M\) of these bodies and inversely
proportional to the square of their geographical distance \(D\). \(G\)
is a constant and as such of no major concern.

\end{frame}

\begin{frame}{Simple gravity model}

The underlying idea of a traditional gravity model, shown for
international trade, is equally simple:

\[
X_{ij}=G\frac{Y_{i}^{\beta_{1}}Y_{j}^{\beta_{2}}}{D_{ij}^{\beta_{3}}}.
\]

The trade flow \(X\) is explained by \(Y_{i}\) and \(Y_{j}\) that are
the masses of the exporting and importing country (e.g.~the GDP) and
\(D_{ij}\) that is the distance between the countries.

\end{frame}

\begin{frame}{Simple gravity model}

\LARGE{This is also used to study urban policies and migration flows!}

\end{frame}

\begin{frame}{Simple gravity model}

Dummy variables such as common borders \(contig\) or regional trade
agreements \(rta\) can be added to the model. Let \(t_{ij}\) be the
transaction cost defined as:

\[
t_{ij}= D_{ij} \exp(contig_{ij} + rta_{ij})
\]

\end{frame}

\begin{frame}{Simple gravity model}

So that the model with friction becomes:

\[
X_{ij}=G\frac{Y_{i}^{\beta_{1}}Y_{j}^{\beta_{2}}}{t_{ij}^{\beta_{3}}}.
\]

\end{frame}

\begin{frame}{Simple gravity model}

A logarithmic operator can be applied to form a log-linear model and use
a standard estimation methods such as OLS:

\[
\log X_{ij}=\beta_{0}\log G +\beta_{1}\log Y_{i}+\beta_{2}\log Y_{j}+\beta_{3}\log D_{ij}+\beta_{4}contig_{ij}+\beta_{5}rta_{ij}
\]

\end{frame}

\section{Trade barriers model}\label{trade-barriers-model}

\begin{frame}{Trade barriers model}

Basically the model proposes that the exports \(X_{ij}\) from \(i\) to
\(j\) are determined by the supply factors in \(i\), \(Y_{i}\), and the
demand factors in \(j\), \(Y_{j}\), as well as the transaction costs
\(t_{ij}\).

Next to information on bilateral partners \(i\) and \(j\), information
on the rest of the world is included in the gravity model with
\(Y=\sum_{i} Y_{i}= \sum_{j} Y_{j}\) that represents the worldwide sum
of incomes (e.g.~the world's GDP).

\end{frame}

\begin{frame}{Trade barriers model}

In this model \(\sigma\) represents the elasticity of substitution
between all goods. A key assumption is to take a fixed value
\(\sigma > 1\) in order to account for the preference for a variation of
goods (e.g.~in this model goods can be replaced for other similar
goods).

\end{frame}

\begin{frame}{Trade barriers model}

The Multilateral Resistance terms are included via the terms \(P\),
Inward Multilateral Resistance, and \(\Pi\), Outward Multilateral
Resistance.

The Inward Multilateral Resistance \(P_i\) is a function of the
transaction costs of \(i\) to all trade partners \(j\).

The Outward Multilateral Resistance \(\Pi_{j}\) is a function of the
transaction costs of \(j\) to all trade partners \(i\) and their demand.

The Multilateral Resistance terms dependent on each other. Hence, the
estimation of structural gravity models becomes \emph{complex}.

\end{frame}

\begin{frame}{Trade barriers model}

\begin{figure}
\centering
\includegraphics{pikachu.jpg}
\caption{What?}
\end{figure}

\end{frame}

\begin{frame}{Trade barriers model}

The econometric literature proposes the Multilateral Resistance model
defined by the equations:

\[
X_{ij}=\frac{Y_{i}Y_{j}}{Y}\frac{t_{ij}^{1-\sigma}}{P_{j}^{1-\sigma}\Pi_{i}^{1-\sigma}}
\] with \[
P_{i}^{1-\sigma}=\sum_{j}\frac{t_{ij}^{1-\sigma}}{\Pi_{j}^{1-\sigma}}\frac{Y_{j}}{Y};\:\Pi_{j}^{1-\sigma}=\sum_{i}\frac{t_{ij}^{1-\sigma}}{P_{i}^{1-\sigma}}\frac{Y_{i}}{Y}
\]

\end{frame}

\begin{frame}{Trade barriers model}

\begin{figure}
\centering
\includegraphics{raichu.jpg}
\caption{Again, what?}
\end{figure}

\end{frame}

\section{Model estimation}\label{model-estimation}

\begin{frame}{Model estimation}

To estimate gravity equations you need a square dataset including
bilateral flows defined by the argument dependent\_variable, a distance
measure defined by the argument distance that is the key regressor, and
other potential influences (e.g.~contiguity and common currency) given
as a vector in additional\_regressors are required.

Some estimation methods require ISO codes or similar of type character
variables to compute particular country effects. Make sure the origin
and destination codes are of type ``character''.

\end{frame}

\begin{frame}{Model estimation}

The rule of thumb for regressors or independent variables consists in:

\begin{itemize}
\tightlist
\item
  All dummy variables should be of type numeric (0/1).
\item
  If an independent variable is defined as a ratio, it should be logged.
\end{itemize}

The user should perform some data cleaning beforehand to remove
observations that contain entries that can distort estimates,
notwithstanding the functions provided within gravity package will
remove zero flows and distances.

\end{frame}

\section{Examples}\label{examples}

\begin{frame}{Double Demeaning}

Double Demeaning subtracts importer and exporter averages on the left
and right hand side of the respective gravity equation, and all
unilateral influences including the Multilateral Resistance terms
vanish.

Therefore, no unilateral variables may be added as independent variables
for the estimation.

\end{frame}

\begin{frame}{Double Demeaning}

Our ddm function first logs the dependent variable and the distance
variable.

Afterwards, the dependent and independent variables are transformed in
the following way (exemplary shown for trade flows, \(X_{ij}\)): \[
(\log X_{ij})_{\text{DDM}} = (\log X_{ij}) - (\log X_{ij})_{\text{Origin Mean}} - (\log X_{ij})_{\text{Destination Mean}} + (\log X_{ij})_{\text{Mean}}.
\]

\end{frame}

\begin{frame}{Double Demeaning}

One subtracts the mean value for the origin country and the mean value
for the destination country and adds the overall mean value to the
logged trade flows.

This procedure is repeated for all dependent and independent variables.
The transformed variables are then used for the estimation.

\end{frame}

\begin{frame}[fragile]{Double Demeaning}

An example of how to apply the function ddm to an example dataset in
gravity and the resulting output is shown in the following:

\begin{Shaded}
\begin{Highlighting}[]
\KeywordTok{p_load}\NormalTok{(gravity)}

\NormalTok{fit <-}\StringTok{ }\KeywordTok{ddm}\NormalTok{(}
    \DataTypeTok{dependent_variable =} \StringTok{"flow"}\NormalTok{,}
    \DataTypeTok{distance =} \StringTok{"distw"}\NormalTok{,}
    \DataTypeTok{additional_regressors =} \KeywordTok{c}\NormalTok{(}\StringTok{"rta"}\NormalTok{, }\StringTok{"comcur"}\NormalTok{, }\StringTok{"contig"}\NormalTok{),}
    \DataTypeTok{code_origin =} \StringTok{"iso_o"}\NormalTok{,}
    \DataTypeTok{code_destination =} \StringTok{"iso_d"}\NormalTok{,}
    \DataTypeTok{data =}\NormalTok{ gravity_no_zeros}
\NormalTok{  )}
\end{Highlighting}
\end{Shaded}

\end{frame}

\begin{frame}[fragile]{Double Demeaning}

The package returns lm or glm objects instead of summaries. Doing that
allows to use our functions in conjunction with broom or other packages,
for example:

\begin{Shaded}
\begin{Highlighting}[]
\KeywordTok{p_load}\NormalTok{(dplyr, broom)}
\KeywordTok{tidy}\NormalTok{(fit)}
\end{Highlighting}
\end{Shaded}

\begin{verbatim}
## # A tibble: 4 x 5
##   term         estimate std.error statistic  p.value
##   <chr>           <dbl>     <dbl>     <dbl>    <dbl>
## 1 dist_log_ddm   -1.60     0.0331    -48.4  0.      
## 2 rta_ddm         0.797    0.0700     11.4  6.54e-30
## 3 comcur_ddm      0.174    0.146       1.19 2.34e- 1
## 4 contig_ddm      1.00     0.120       8.36 6.62e-17
\end{verbatim}

\end{frame}

\begin{frame}[fragile]{Double Demeaning}

\begin{Shaded}
\begin{Highlighting}[]
\KeywordTok{glance}\NormalTok{(fit) }\OperatorTok\StringTok{ }\KeywordTok{select}\NormalTok{(}\KeywordTok{matches}\NormalTok{(}\StringTok{"squared"}\NormalTok{))}
\end{Highlighting}
\end{Shaded}

\begin{verbatim}
## # A tibble: 1 x 2
##   r.squared adj.r.squared
## *     <dbl>         <dbl>
## 1     0.254         0.254
\end{verbatim}

\end{frame}

\begin{frame}[fragile]{Double Demeaning}

How does the internal code work?

\begin{Shaded}
\begin{Highlighting}[]
\NormalTok{ddm <-}\StringTok{ }\ControlFlowTok{function}\NormalTok{(dependent_variable,}
\NormalTok{                distance,}
                \DataTypeTok{additional_regressors =} \OtherTok{NULL}\NormalTok{,}
\NormalTok{                code_origin,}
\NormalTok{                code_destination,}
                \DataTypeTok{robust =} \OtherTok{FALSE}\NormalTok{,}
\NormalTok{                data, ...) \{}
\NormalTok{...}
\end{Highlighting}
\end{Shaded}

\end{frame}

\begin{frame}[fragile]{Double Demeaning}

How does the internal code work?

\begin{Shaded}
\begin{Highlighting}[]
\CommentTok{# Checks}
  \KeywordTok{stopifnot}\NormalTok{(}\KeywordTok{is.data.frame}\NormalTok{(data))}
  \KeywordTok{stopifnot}\NormalTok{(}\KeywordTok{is.logical}\NormalTok{(robust))}
\NormalTok{...}

\CommentTok{# Transforming data, logging distances}
\NormalTok{  d <-}\StringTok{ }\NormalTok{d }\OperatorTok
\StringTok{    }\KeywordTok{mutate}\NormalTok{(}
      \DataTypeTok{dist_log =} \KeywordTok{log}\NormalTok{(}\OperatorTok{!!}\KeywordTok{sym}\NormalTok{(distance))}
\NormalTok{    )}
\NormalTok{...}
\end{Highlighting}
\end{Shaded}

\end{frame}

\begin{frame}[fragile]{Double Demeaning}

How does the internal code work?

\begin{Shaded}
\begin{Highlighting}[]
\CommentTok{# Transforming data, logging flows}
\NormalTok{  d <-}\StringTok{ }\NormalTok{d }\OperatorTok
\StringTok{    }\KeywordTok{mutate}\NormalTok{(}
      \DataTypeTok{y_log =} \KeywordTok{log}\NormalTok{(}\OperatorTok{!!}\KeywordTok{sym}\NormalTok{(dependent_variable))}
\NormalTok{    )}
\NormalTok{...}
\end{Highlighting}
\end{Shaded}

\end{frame}

\begin{frame}[fragile]{Double Demeaning}

How does the internal code work?

\begin{Shaded}
\begin{Highlighting}[]
\CommentTok{# Substracting the means}
\NormalTok{  d <-}\StringTok{ }\NormalTok{d }\OperatorTok
\StringTok{    }\KeywordTok{mutate}\NormalTok{(}
      \DataTypeTok{y_log_ddm =} \OperatorTok{!!}\KeywordTok{sym}\NormalTok{(}\StringTok{"y_log"}\NormalTok{),}
      \DataTypeTok{dist_log_ddm =} \OperatorTok{!!}\KeywordTok{sym}\NormalTok{(}\StringTok{"dist_log"}\NormalTok{)}
\NormalTok{    ) }\OperatorTok
\StringTok{    }\KeywordTok{group_by}\NormalTok{(}\OperatorTok{!!}\KeywordTok{sym}\NormalTok{(code_origin), }\DataTypeTok{add =} \OtherTok{FALSE}\NormalTok{) }\OperatorTok
\StringTok{    }\KeywordTok{mutate}\NormalTok{(}
      \DataTypeTok{ym1 =} \KeywordTok{mean}\NormalTok{(}\OperatorTok{!!}\KeywordTok{sym}\NormalTok{(}\StringTok{"y_log_ddm"}\NormalTok{), }\DataTypeTok{na.rm =} \OtherTok{TRUE}\NormalTok{),}
      \DataTypeTok{dm1 =} \KeywordTok{mean}\NormalTok{(}\OperatorTok{!!}\KeywordTok{sym}\NormalTok{(}\StringTok{"dist_log_ddm"}\NormalTok{), }\DataTypeTok{na.rm =} \OtherTok{TRUE}\NormalTok{)}
\NormalTok{    ) }\OperatorTok
\NormalTok{...}
\end{Highlighting}
\end{Shaded}

\end{frame}

\section{Code and documentation}\label{code-and-documentation}

\begin{frame}{Code and documentation}

\LARGE{github.com/pachamaltese/gravity}

\LARGE{pacha.hk/gravity}

\end{frame}

\section{Questions?}\label{questions}

\begin{frame}{Questions?}

\LARGE{Thanks for your attention!}

\end{frame}

\section{References}\label{references}

\begin{frame}{References}

\end{frame}

\begin{frame}[allowframebreaks]{}
\bibliographytrue
\bibliography{references.bib}
\end{frame}

\end{document}
